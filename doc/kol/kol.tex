\documentclass[10pt, xcolor=x11names]{beamer}
\usecolortheme{seagull}
\useoutertheme{infolines}
\usefonttheme[onlymath]{serif}
\setbeamertemplate{headline}[default]
\setbeamertemplate{navigation symbols}{}
\mode<beamer>{\setbeamertemplate{blocks}[rounded][shadow=true]}
\setbeamercovered{transparent}
\setbeamercolor{block body example}{fg=blue, bg=black!20}

\usepackage[utf8]{inputenc}
\usepackage[german]{babel}
\usepackage[]{csquotes}
\usepackage{amsmath}
\usepackage{tikz, wasysym}
\usepackage{graphicx}
\usetikzlibrary{automata,positioning}
\usepackage{hyperref}

%\usepackage{amsfonts}
%\usepackage{amssymb}
%\usepackage{makeidx}
%\usepackage{graphicx}


\usepackage{hyperref}
\author{Sven Fiergolla}
\title[Colloquium]{Improving Run Length Encoding through preprocessing}
\subtitle[short version]{}
\date{14. Januar 2020}
%\institute[Uni Trier]{Universität Trier}
%\logo{\includegraphics[scale=.25]{unilogo.pdf}}

\begin{document}
	\frame{\maketitle}
	\frame{\frametitle{}
	\tableofcontents
	}



\section{Introduction}	
\frame{\frametitle{Introduction - A Bit of History}
	  \begin{itemize}
		\item rise of multimedia
		\item rise of the World Wide Web
		\item ever increasing data transfer
	\end{itemize}

\medskip 

  \begin{itemize}
	\pause \item compress to save storage space \& to handle new types and volumes of data
\end{itemize}	
}

\frame{\frametitle{Introduction - The Situation Today}
	\begin{itemize}
		\item burst of sensors and IoT
		\item massive and rapid increasing data transfer
	\end{itemize}
	
	\medskip 
	
	\begin{itemize}
		\pause \item compress to lower transmission cost / time
		\item compress to handle increasing resolution, fidelity, dynamic range
		\item compression for cold archiving
	\end{itemize}	
}

\section{Basics}
\frame{\frametitle{Basics of Compression}
	\begin{itemize}
		\item Non random data contains redundant information
		\item Compression is about pattern or structure identification and exploitation
		\pause \item No algorithm can compress all possible data of a given length, even by one byte (Kolmogorov Complexity)
	\end{itemize}	
}
\end{document}
