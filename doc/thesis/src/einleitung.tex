%% Einleitung.tex
%% $Id: einleitung.tex 61 2012-05-03 13:58:03Z bless $
%%

\chapter{Introduction}
\label{ch:Introduction}
%% ==============================
%% ==============================
\section{Motivation}
%% ==============================
\label{ch:Introduction:sec:Motivation}
\par{
In the last decades, digital data transfer became available everywhere and to everyone. This rise of digital data urges the need for data compression techniques or improvements on existing ones. Run-length encoding \cite{rle-patent} (abbreviated as RLE) is a simple coding scheme that performs lossless data compression. RLE compression simply represents the consecutive, identical symbols of a string with a run, usually denoted by $\sigma^i$, where $\sigma$ is an alphabet symbol and $i$ is its number of repetitions. To give an example, the string \emph{aaaabbaaabbbba} can be compressed into RLE format as  $ a^{4}b^{2}a^{3}b^{4}a^{1}$ . Thanks to its simplicity it is still being used in several areas like fax transmission, where RLE compression is combined with other techniques into Modified Huffman Coding \cite{fax-rle} described in Section \ref{ch:Principles of compression:sec:Huffman Coding}. Most fax documents are typically simple texts on a white background, RLE compression is particularly suitable for fax and often achieves good compression ratios.
}
%% ==============================
\section{Problem statement}
%% ==============================
\label{ch:Introduction:sec:Problem statement}
\par{
Some strings like \emph{aaaabbbb} achieve a very good compression rate because the string only has two different characters and they repeat more than twice. Therefore it can be compressed to $a^4b^4$ so from 8 byte down to 4 bytes if you encode it properly. On the other hand, if the input is highly mixed characters with few or no repetitions at all like \emph{abcdefgh}, the run length encoding of the string is $a^1b^1c^1d^1e^1f^1g^1h^1$ which needs up to 16 bytes depending on the implementation. So the inherent problem with run length encoding is obviously the possible explosion in size, due to missing repetitions in the input string. Expanding the string to twice the original size is rather undesirable worst case behavior for a compression algorithm so one has to make sure the input data is fitted for RLE as compression scheme. One goal is to improve the compression ratio on data currently not suited for run length encoding and perform better than the originally proposed RLE, in order for it to work on arbitrary data. Another goal should be to minimize the increase in size in the worst case scenario.}

%% ==============================
\section{Main Objective}
%% ==============================
\label{ch:Introduction:sec:Main Objective}
\par{
The main objectives that derives from the problem statement, is to achieve an improved compression ratio compared to regular run length encoding on strings or files that are currently not suited for the method. Additionally it is desirable to further increase its performance in cases it is already reasonable. To unify the measurements, the compression ratio is calculated by encoding all files listed in the Calgary corpus which will be presented in Section \ref{tab:t05 The Calgary Corpus}. Since most improvements like permutations on the input, for example a reversible Burros-Wheeler-Transformation to increase the number of consecutive symbols or a different way of reading the byte stream take quite some time, encoding and decoding speed will decrease with increasing preprocessing effort.
}
%% ==============================
\section{Structure of this work}
%% ==============================
\label{ch:Intoduction:sec:Structure}
\par{
This work is structured into a first introduction into the basics of compression and the applied methods known to this discipline which will be used further on and an analysis of the current state of the art. Then, the conceptual design is depicted with a following analysis of the results. Afterwards, the implementation of the algorithms are described and the work as a whole is evaluated with a short closing discussion.  
}
%%% Local Variables: 
%%% mode: latex
%%% TeX-master: "thesis"
%%% End: 
