%% analyse.tex
%% $Id: analyse.tex 61 2012-05-03 13:58:03Z bless $

\chapter{Analysis}
\label{ch:Analysis}
%% ==============================

\par{
The following chapter contains a detailed analysis of the problem and some fundamental requirements for the algorithm. To have a comparison to some extent, the initial performance analysis was performed on the Calgary Corpus but the results of the unmodified run length coding algorithm where very underwhelming as expected.
}

%% ==============================
\section{Initial Findings}
%% ==============================
\label{ch:Analysis:sec:Initial Findings}
\par{
 Originally developed and used on black and white pallet images containing only two values, on a binary level so to speak, we want to use it for rather arbitrary data, mostly text. The initial algorithm is not suited for that because continuous text as binary representation does not contain runs of any kind, which could be compressed. The ASCII representation of the letter 'e' which is the most common in general English, has the value 130 or '01100101' as 7-bit ASCII or '001100101' as byte value of the UTF-8 representation. As you can see, there are no runs of a considerable size and this is the case for most printable characters as they all have a value between 32 and 127 (or 255 for the extended ASCII). Applied to the Calgary Corpus in this simple implementation, there should be an increase in size expected or \textit{negative compression} as one might say.
}

\begin{center}
	\begin{tabular}[p]{l|r|r}
		\label{tab:t4 simple run length eval}
		
		bits per rle number &  expansion ratio & bits per symbol in $\frac{bits}{symbol}$\\
		\hline
		8 & 3.298046423741734& 26.38437138993387 \\
		7 & 2.8897459975751163& 23.11796798060093\\
		6 & 2.4839251325134675& 19.87140106010774 \\
		5 & 2.0825541259578895 & 16.66043300766311\\
		4 & 1.6895640359371056 & 13.516512287496845\\
		3 & 1.3130541262757818 & 10.504433010206254\\
		2 & 1.04555716691706 & 8.36445733533648 \\
	\end{tabular}
\end{center}

\par{
The results in Table \ref{tab:t4 simple run length eval} depict the anticipated, an increase in size regardless of the amount of bits used to encode a run. By using 8 bits to encode a single run, in the worst case scenario a byte which is only alternating values like 01010101, would expand to 8 bytes, all encoding a run of length 1.
}

\par{
RLE is also applicable on a byte level, because there should be repetitions of any kind like consecutive letters or line endings (EOL). Byte level RLE encodes runs of identical byte values, ignoring individual bits and word boundaries. The most common byte level RLE scheme encodes runs of bytes into 2-byte packets. The first byte contains the run count of 1 to 256, and the second byte contains the value of the byte run. If a run exceeds a count of 256, it has to be encoded twice, one with count 256 and one with any further runs.
}

\begin{center}
	\begin{tabular}[p]{l|r|r}
		\label{tab:t5 run length eval}
		
		bits per rle number &  ratio in \% & bits per symbol in $\frac{bits}{symbol}$\\
		\hline
		8 & 165 & 13.2 \\
		7 & 154 & 12.38\\
		6 & 144 & 11.57 \\
		5 & 134 & 10.77\\
		4 & 125 & 10.00\\
		3 & 116 & 9.29\\
		2 & 109 & 8.74 \\
	\end{tabular}
\end{center}


\par{
 However after some analysis of the corpus data, it was shown that most runs had a value of one and almost no runs larger that 4 occurred, which lead to the conclusion, two bit for the run count should be plenty, which is also shown in Table \ref{tab:t5 run length eval}.
 Even with a run size of just two bits, there is still a increase in size of about 9\% and uses 8.74 $\frac{bits}{symbol}$. This is still useful as a kind of a base line.
Interestingly the binary implementation performs better on 2 bits per RLE number (4 \% increase in size) than the byte implementation (9 \% increase in size) but also worse with a higher amount of bits per run, where it expands the data to more than triple in size. It is unclear which kind of implementation will profit most of preprocessing, so both will be further analyzed.
}

\par{
If we take a more detailed look, we can see that while most files expand with larger RLE numbers, some files have their minimum size when encoded with higher RLE numbers of up to 7 bit. With the simple binary based RLE, almost all files of the Calgary Corpus expand linear related to the amount of bits used for the encoding. The file \textit{pic} decreases in size until 7 bits per RLE number used to a sizes of just $19.5 \%$ of its original size with only  $1.56 \: \frac{bits}{symbol}$ while the other files just doubled or even tripled in size. Using the byte wise operating RLE we see a similar result but not as decent with $27.2\%$ of its original size using $2.17 \: \frac{bits}{symbol}$ using 6 bits per run. The benefit of the byte wise RLE is the better worst case performance of $1.5$ up to $1.7$ times the original size.
}


%% ==============================
\section{Possible Improvements by Preprocessing}
%% ==============================
\label{ch:Analysis:sec:Improvements by Preprocessing}

The broad idea of preprocessing is to manipulate the input data in a way that results in data which can be compressed more efficiently than the original data. This can be done in various ways, some of them will be explored in greater detail to find out if it is implementable or not. One way of doing so is a Burrows-Wheeler-Transformation.

\subsection{Burrows-Wheeler-Transformation}
\label{ch:Analysis:sec:Improvements by Preprocessing:subSec:bwt}
\par{
To understand how a Burrows-Wheeler-Transformation improves the effectiveness of compression, consider the effect in a common word in English text. Examine the letter ‘t’ in the word ‘the’, in an input string holding multiple instances of this word.
Sorting all rotations of a string results in all rotations starting with ‘he ’ will be sorted together and most of them are going to to end in the letter ‘t’. This implies that the transformed string L has a large number of the letter t, combined with some other characters, such as space, ‘s’, ‘T’, and ‘S’. This is true for all characters, so any substring of L is likely to contain a large number of a some distinct characters. \enquote{The overall effect is that the probability that given character $c$ will occur at a given
point in L is very high if $c$ occurs near that point in L, and is low otherwise.} \cite{Burrows94}
}

\par{
It is obvious that this should always improve the performance of byte level RLE because the transformation is taking place at character level but it should not effect the binary implementations. 
}

\subsection{Vertical byte reading}
\par{
Instead of performing compute intense operation on the data, we could also interpret the data in a different way and apply the original run length encoding on binary data. By reading the data in chunks of a fixed size, it is possible to read all most significant bits of all bytes, then the second most significant bits of all bytes and so on. This interpretation results in longer runs as shown in the example below.
}

TODO: example

\par{
It is clear that simply a different way of reading the input does not compress the actual data, instead it enables a better application of existing compression. Without any further action, no performance improvements will be made.
}

\subsection{Byte remapping}
\par{
One idea might be a dynamic byte remapping as the input data is read in parts. Some sections have more specific characters or bytes than others, or this idea can be applied to the whole file. This way the values are not alternating in the whole range of 0 to 255 but rather in a smaller subset and the most frequent ones should be mapped to the smallest values.
}

TODO: Example

\par{

}

\subsection{Combined approaches}
\par{
The idea of combining different compression methods into a superior method is not new and was also performed on RLE as mentioned in Section \ref{ch:Principles of compression:sec:Run Length Encoding:subSec:History}. While the idea of encoding the RLE numbers with Huffman codes is already known and analyzed well, the vertical byte reading enables new approaches, even more in combination with the idea of byte remapping.
}


%% ==============================
\section{Summary}
%% ==============================
\label{ch:Analyse:sec:Summary}

Am Ende sollten ggf. die wichtigsten Ergebnisse nochmal in \emph{einem}
kurzen Absatz zusammengefasst werden.

%%% Local Variables: 
%%% mode: latex
%%% TeX-master: "thesis"
%%% End: 
