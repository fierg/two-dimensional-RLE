%% analyse.tex
%% $Id: analyse.tex 61 2012-05-03 13:58:03Z bless $

\chapter{Analysis}
\label{ch:Analysis}
%% ==============================
In diesem Kapitel sollen zunächst das zu lösende Problem
sowie die Anforderungen und die Randbedingungen 
einer Lösung\index{Lösung} beschrieben werden (eine präzisierte Aufgabenstellung\index{Aufgabenstellung}).
\ldots

%% ==============================
\section{Prerequisites}
%% ==============================
\label{ch:Analysis:sec:Prerequisites}
Anforderungen und Randbedingungen\index{Randbedingungen} \ldots

%% ==============================
\section{Initial Findings}
%% ==============================
\label{ch:Analysis:sec:Initial Findings}
- no matrix representation but chunks of type byteArray\\
- static encoding difficulties\\
\ldots

%% ==============================
\section{Improvements by Preprocessing}
%% ==============================
\label{ch:Analysis:sec:Improvements by Preprocessing}
- First improvements due to byte remapping\\
- burrows wheeler transformation\\
\ldots

%% ==============================
\section{Further Improvements}
%% ==============================
\label{ch:Analysis:sec:Further Improvements}
- combining different compression techniques\\
\ldots


%% ==============================
\section{Summarization}
%% ==============================
\label{ch:Analyse:sec:Summarization}

Am Ende sollten ggf. die wichtigsten Ergebnisse nochmal in \emph{einem}
kurzen Absatz zusammengefasst werden.

%%% Local Variables: 
%%% mode: latex
%%% TeX-master: "thesis"
%%% End: 
