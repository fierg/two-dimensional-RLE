%% grundlagen.tex
%% $Id: grundlagen.tex 61 2012-05-03 13:58:03Z bless $
%%

\chapter{Basic principles of compression}
\label{ch:Basic principles of compression}
%% ==============================

To understand compression one first hast to understand some basic pricipels of information theory like Entropy and different approaches to compress different types of data with different encoding and entropy. I will also show the key differences between probability coding and dictionary coding and a few comments on lossy compression.

%% ==============================

\section{Compression and Encoding fundamentals}
%% ==============================
\label{ch:Basic principles of compression:sec:Compression}

...
\subsection{Entropy and Unit of Compression}
\subsection{Probability Coding}
\subsection{Dictionary coding}
\subsection{Irreversible Compression}


%% ==============================
\section{Run Length Encoding}
%% ==============================
\label{ch:Grundlagen:sec:Run Length Encoding}

...

%% ==============================
\section{Huffman Coding}
%% ==============================
\label{ch:Grundlagen:sec:Huffman Coding}

...

%% ==============================
\section{State of the Art}
%% ==============================
\label{ch:Grundlagen:sec:SOTA}
Die Literaturrecherche soll so vollständig wie möglich sein und bereits existierende relevante Ansätze (Verwandte Arbeiten / State of the Art / Stand der Technik) beschreiben bzw. kurz vorstellen.
Es soll aufgezeigt werden, wo diese Ansätze Defizite aufweisen oder nicht anwendbar sind, z.B. weil sie von anderen Umgebungen oder Voraussetzungen ausgehen.

Je nach Art der Abschlussarbeit kann es auch sinnvoll sein, diesen Abschnitt in die Einleitung zu integrieren oder als eigenes Kapitel aufzuführen.

Beispiel, wie mit LaTeX zitiert werden kann: \cite{TB98,JSAC96,qosr}


State of the art compression
- techniques\\
- use cases\\

%%% Local Variables: 
%%% mode: latex
%%% TeX-master: "thesis"
%%% End: 
