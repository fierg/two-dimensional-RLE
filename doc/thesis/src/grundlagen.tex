%% grundlagen.tex
%% $Id: grundlagen.tex 61 2012-05-03 13:58:03Z bless $
%%

\chapter{Principles of compression}
\label{ch:Principles of compression}
%% ==============================
\par{
Throughout this paper we fix the finite non-empty alphabet $\Sigma$ and assume that $\Sigma$ is equipped with a linear order $\leq$. A word $w$ is a sequence $a_1,...,a_n$ of letters $a_i \in \Sigma$, $1 \leq i \leq n$. The set of all such sequences is denoted by $\Sigma^*$ which is the free monoid over $\Sigma$, with concatenation as composition and with the empty word $\varepsilon$ as neutral element. The set $\Sigma^+ = \Sigma^*$ \textbackslash $\: \varepsilon$ consists of all non-empty words. Each word induces a labeling function $\lambda(w)$ as a labeled linear order where position $i$ of $w$ is labeled by $a_i \in \Sigma$ and we write $\lambda_w (i) = a_i$.

If, for words $u, \: v$,  $u = v$ or $u$ is lexicographically smaller than $v$ we write $u \leq v$ in the context of the order of letters.  The ordered conjugacy class of a word $w \in \Sigma^n$ as $[w] = (w_q,...,w_n)$. The unordered conjugacy class of $w$ is the set $\{w_1, ... , w_n\}$. Whenever there is no confusion we can abuse the notation and also denote the unordered conjugacy class for $w$ s $[w]$. This is for example the case if $w$ is somehow distingushed in it conjugacy class for example if $w$ is a Lyndon word. A \emph{Lyndon word} is the unique minimal element within its conjugacy class. More formally, let $[w] = (w,w_2,...w_n)$, then $w \: \in \Sigma^+$ is a Lyndon word if $w<w_i$ for all $i \: \in \{2,...,n\}$.
}

%% ==============================
\section{Compression and Encoding fundamentals}
%% ==============================
\label{ch:Principles of compression:sec:Compression}
\par{
	The basic idea of compression is to remove redundancy in data since all non random data contains redundant information. Pattern or structure identification and exploitation enables storing the original data in less space. Compression can be broken down into two broad categories: Lossless and lossy compression. Lossless compression makes it possible to reproduce the original data exactly while lossy compression allows the some degradation in the encoded data to gain even higher compression at the cost of some of the original information. To understand compression, one first has to understand some basic principles of information theory like entropy and different approaches to compress different types of data with different encoding. We will also show the key differences between probability coding and dictionary coding.}

\subsection{Information Theory and Entropy}
\par{
As Shannon described his analysis about the English language \cite{entropy}, he used the term entropy closely aligned with its definition in classical physics, where it is defined as the disorder of a system. Specifically speaking, it is assumed that a system has a set of states of which it can be in and it exists a probability distribution over those states. Shannon then defines the entropy as:
\[
H(S) = \sum_{s \in S} P(s) \: i(s)
\]
where $S$ describes all possible states, $P(s)$ is the likelihood of the system being in state $s$. So generally speaking it means that evenly distributed probabilities imply a higher entropy and vice versa.
It also implies that, given a source of information $S$, its average information content per message from $S$ is also described by this formula.}

\par{
In addition to that, Shannon defined the term self information $i(s)$ as:
\[
i(S) = log_{2} \: \frac{1}{P(s)}
\]
indicating that with higher probability of a state, less information can be contained. As an example, the statement \enquote{The criminal is smaller than 2 meters.} is very likely but doesn't contain much information, whereas the statement \enquote{The criminal is larger than 2 meters.} is not very likely but contains more information. With these definitions in mind, we can analyze some properties for the English language from an information theory point of view.
}

\par{
Combining those approaches, you will find that $\sum$ being a finite alphabet, and $P(s_i)$ describing the likelihood of $s_i$ and $i(s_i)$ its self information, $i(s_i)$ also describes the amount of bits needed for this symbol, therefore its $log_2$.
\[
	H(\sum) = \sum_{i = 1}^{n} P(s_i) \: \cdot \: log_{2} \: \frac{1}{P(s_i) }
\]
So the entropy 	also describes the theoretical amount of bits needed to persist a message from $\sum$, calculated in $bps$ as bits per symbol.

\begin{table}
	\centering
\begin{tabular}[p]{l|l|l|l|l|l|r}
	& $P(a)$ & $P(b)$ & $P(c)$ & $P(d)$ & $P(e)$ & $H$ \\
	\hline
	1. & 0.2 & 0.2 & 0.2 & 0.2 & 0.2 & 2.322 \\
	2. & 0.94 & 0.01 & 0.01 & 0.01 & 0.01 & 0.322
	\label{tab:heisetabelle}
\end{tabular}
	\caption{Entropy in relation to the Probability of Symbols}
\end{table}


TODO: describe some relations between probability and entropy of information source
}
\subsection{General Analysis}
\par{
To evaluate the efficiency of a specific compression technique, we have to determine how much information the raw data contains. We are looking at ASCII encoded English language sentences as examples for text compression. There have been broad analysis of ASCII entropy, consisting of 96 different printable symbols for the English language, generating approaches for calculating the entropy \cite{entropy-fernau}.
}

\par {
If we assume a 96 symbol alphabet and a uniform probability distribution, we find an entropy of $log_2 (96) = 6.6 \: bps$. An empirically distribution generated by text analysis, the entropy comes down to $4.5 \: bps$. Using an encoding which encodes each symbol separately instead of the whole input at once, like the Huffman Coding, we can achieve an entropy of $4.7 \: bps$ which is only slightly worse than the assumed entropy. By changing the assumption to blocks of symbols of length 8, we get $96^8$ different blocks. Although the probability distribution of the English language implies an entropy as low as $1.3 \: bps$ which leads to a possible reduction of symbols to 3 printable characters without generating longer texts. Consulting newer sources we will find that, up to 500 character long symbol based text analysis, with around $20.3 \: Mio.$ different symbols, results in an entropy of around $1,58 \: bps$ \cite{entropy-new}. This gives a vague limit of how much compression we can theoretically expect from English text.}


\subsection{Probabilistic Coding}
\par{
The general idea behind Probability Coding is to analyze probabilities for messages and encode them in variable length bit strings according to their probability. This way, messages that are more likely and will repeat more often can be encoded in a smaller bit string representation. The generation of those probability distributions is considered part of the analyzer module by the algorithm and will be discussed later on in Section \ref{ch:Conceptual Design}. Probability coding can be further distinguished into variable and fixed unique coding, which will represent each message with a bit string with an amount $n$ of bits $n \in N$, for example Huffman coding as type of prefix coding which will be explained in greater detail in section \ref{ch:Principles of compression:sec:Huffman Coding}. In contrast to Huffman coding there is also arithmetic coding which can represent a set of messages as a single floating point number $q$ where $0.0 < q < 1.0$. By encoding them in nestend intervals in $[0,1]$ determined by a probability distribution, they can \enquote{fuse} messages together and need less space to represent a set of messages than encoding every message separately. Arithmetic coding is usually stronger than Huffman coding but also slower. In 2014 a newer probabilistic approach was produced by Jarek Duda called Asymmetric Numeral Systems (ANS) \cite{DBLP:journals/corr/Duda13}. This method encodes a string of symbols into a single natural number, combining the strength of Arithmetic coding with the speed of Huffman coding.
}

\subsection{Dictionary Coding}
\par{
Dictionary Coding is best suited for data with a small amount of repeating patterns like a text containing repeated words. In this case it is very effective to save the patterns just once and refer to them if they are used later on in the text. If we know a lot about the text itself in advance, a static dictionary is sufficient but in general we want to compress an arbitrary text. This requires a more general approach as used by the well known LZ77 and LZ88 algorithms described by Jacob Ziv and Abraham Lempel \cite{lz}. They use a so called sliding window principle, a buffer moving across the input in the case of LZ77 or a dynamic dictionary in the implementation of LZ78. Both of them are still used in modified versions in real world applications as described in section \ref{ch:Principles of compression:sec:SOTA}, although mostly the are implemented in newer variants or derivatives.
}
\par{
The main difference between probabilistic coding and dictionary coding is, that the so far presented probability based methods like RLE or Huffman Coding are working on single characters or bytes whereas the dictionary based methods encode groups of characters of varying length. This is a clear advantage in terms of compression performance on every day data because of its repeating patterns most textual data has, as shown in Section \ref{ch:Analysis:sec:Initial Findings}.
}


\subsection{Irreversible Compression}
\par{
Irreversible Compression or also called \enquote{Lossy Compression} is a type of compression which looses information in the compression process and is therefore not completely reversible, hence the name. There are a lot of use-cases for these type of compression algorithms, mostly used for images, video and audio files. These files typically contain information almost not perceptible for an average human like really small color differences between two pixels of an image or very high frequencies in an audio file. By using approximations to store the information and accept the loss of some information while retaining most of it, lossy image compression algorithms can get twice the compression performance compared to lossless algorithms. Due to the fact that most lossy compression techniques are not suited for text compression which is the main use case for this work, we will not elaborate any further on this topic.
}


%% ==============================
\section{Run Length Encoding}
%% ==============================
\label{ch:Principles of compression:sec:Run Length Encoding}
\par{
	Run length coding might count as the simplest probability coding scheme that makes use of context and repeating occurrences as described in the Section \ref{ch:Introduction:sec:Motivation}. While the example made earlier was in textual representation, it is best suited and mostly used for pallet based bitmap images \cite{palette-image} such as fax transmissions or computer icons.
}

\subsection{The History}
\label{ch:Principles of compression:sec:Run Length Encoding:subSec:History}
\par{
 The ITU-T T4 (Group 3) standard for Facsimile (fax) machines \cite{ITU} is still in force for all devices used over regular phone lines. Each transmission sends a black and white image, where each pixel is called a \textit{pel} and with a horizontal resolution of $8.05 \: \frac{pels}{mm}$ and the vertical resolution depending on the mode. To encode each sequence of black and white pixels, the T4 standard uses RLE to encode each sequence of black and white pixels and since there are only two values, only the run length itself has to be encoded. It is assumed that the first run is always a white run, so there is a dummy white pel at the beginning of each sequence. For example, the sequence $bbbbwwbbbbb$ can be encoded as 1,4,2,5 with the leading white dummy pixel.
}
\par{
To further reduce the RLE encoded sequence, a probability coding procedure like Huffman coding (explained in section \ref{ch:Principles of compression:sec:Huffman Coding}) can be used do code the sequence, since shorter run lengths are generally more common than very long runs, which is also done by the T4 standard. Based on a broad analysis of the average run lengths counted, the T4 defined a set of static Huffman codes, using different codes for the black and white pixels. This way for example 20 consecutive white pels  are a white run with length 20 an so the message can be encoded and transmitted as the code-word \enquote{0001000}, requiring only 7 bit instead of 20, seen in row 6 column 2 in table \ref{tab:t1:static huffman codes}. Since a fax typically contains more white, the white runs are more frequent and thus get shorter code-words which is explained in greater detail in section \ref{ch:Principles of compression:sec:Huffman Coding}. Since the appearance of text usually only needs a few consecutive black pels, short black runs of length 2 or 3 appear to be the most frequent seen runs so to save space they are encoded in the overall shortest codes. Runs of more than 64 have to be encoded in multiple Huffman codes, for example a run of 150 has to be encoded with the Huffman code of 128 followed by the code for 22.
}
\begin{table}
	\centering
	\begin{tabular}[p]{r|l|l}
		run length &  white codeword & black codeword\\
		\hline
		0 &  00110101 & 0000110111\\
		1 & 000111 & 010\\
		2 & 0111 & 11\\
		3 & 1000 & 10\\
		4 & 1011 & 011\\
		... &  & \\
		20 & 0001000 & 00001101000\\
		... & & \\
		64+ & 11011 & 0000001111\\
		128+ & 10010 & 000011001000
		\label{tab:t1:static huffman codes}
	\end{tabular}
	\caption{T4 static Huffman codes}
\end{table}
\par{
By combining RLE with a probabilistic approach, the compression ratio increases because we no longer have to encode a run of length 1 or 2 with a fixed size of bits each instead we can write a recognizable Huffman code of varying length, which will be explored in grater detail later on as well as the decoding of the shown Huffman codes.}

\subsection{Limitations}
\par{
As mentioned in section \ref{ch:Introduction:sec:Problem statement}, run length encoding is rarely used for regular text or continuous tone images because its potentially increase in size, due to non repetitive characters or bytes. A detailed analysis of this issue is performed in section \ref{ch:Analysis:sec:Initial Findings}. Several approaches to mitigate this issue have been implemented in this scope, some of the (e.g. Burrows-Wheeler-Transformation) in multiple variants.
}

\subsection{Run Length Encoding today}
\par{
While plain RLE is still in use by fax transmission or other highly specific tasks, it is mostly implemented in combination with other approaches.
TODO: give more references  \cite{rle-dna}
}
%% ==============================
\section{Prefix Coding}
%% ==============================
\label{ch:Principles of compression:sec:Huffman Coding}
\par{
Prefix codes make use of an untypical idea in computer science. Usually we deal with fixed length codes like 7 bit ASCII or 32 bit Integer representations which map each possible value into that fixed amount of bits. For compression it would be a benefit if we could write codes of variable length. The problem with these varying length codes is that as soon as they are part of a sequence, it becomes very hard to tell where one code word starts and finishes, resulting in ambiguous encoding. For example the set of codes $ C= \{(a,1),(b,01),(c,101),(d,011)\}$ and the encoded message is $1011$ do not generate distinct decodeable code words because there is no way to tell which message was encoded. From now on a code $C$ for a set of messages $S$ is considered to be in the form of $ C= \{(s_1,w_1),(s_2,w_2),...,(s_m,w_w)\}$.
}
\par{
One way to address this problem is by adding extra stop symbols or encoding a length before each code but those just add additional data. Another approach is to use so called prefix codes and make no code which is prefix of another code, which makes them distinct and uniquely decodable as show in the next section.
}

\subsection{Optimal prefix codes - Huffman Coding}
\par{
A Huffman coding is one way of generating a prefix code, which is a uniquely decodable. When no code is prefix of another one, it is always decodable and yields a unique result because once a matching code is read, no other longer could also match. Huffman codes have another property, as they are call
optimal prefix codes. To understand this property we first have to define the average length $l_a$ of a code $C$ as 
\[
l_a(C) = \sum_{(s,w)\in C} \: p(s) \: l(w)
\]
We say that a prefix code is optimal, if $l_a(C)$ is minimized, so there is no other prefix code for a given probability distribution that has a lower average length. The existence of a relation between the average length of a prefix code to the entropy of a set of messages can be shown by making use of the Kraft McMillan Inequality. For a uniquely decodeable code $C$ where $l(w)$ is the length of the code word $C$
\[
\sum_{(s,w)\in C} 2^{-l(w)} \leq 1
\]
And it is also proven that the Huffman algorithm generates optimal prefix codes \cite{compressionIntroduction}. The algorithm	 in general is rather simple and the generation of the prefix codes will be demonstrated using an example. Assume we want to generate a prefix code for the message \enquote{accbccac}. To generate a Huffman tree, first all occurrences of source symbols are counted and then a leaf node with its frequency is added to a priority queue for every symbol. Then, the two nodes with the lowest frequency are removed from the list, combined into one node with the two original as children, which is then added back to the list with the sum of their frequencies. This step is repeated while there are more than one nodes left in the queue. For our example the counts are a = 2, b = 1 and c = 5. So the first two nodes are b and a which are combined and become a new node, then this node is combined with c. To get the actual mapping between a symbol and its code, simply follow the tree from its root to each symbol, the path from the root is its code. For decoding it is required to persists the mapping as well.

TODO: fix numbers on vector
\begin{figure}[h]
	\centering
	\begin{tikzpicture}[->,>=stealth',level/.style={sibling distance = 5cm/#1,
		level distance = 1.5cm}] 
	\node [arn_n] {}
	child{ node [arn_n] { } 
		child{ node [arn_n] {b}}
		child{ node [arn_n] {a}}                            
	}
	child{ node [arn_n] {c}}; 
	\end{tikzpicture}
	\caption{example Huffman Tree with 3 Leaf Nodes} \label{fig:M1:example Huffman tree}
\end{figure}
}
\par{
Decoding is very trivial due to the prefix codes used b this algorithm. Reading bit by bit one can always see if a code is matching one of the mapping and if not, another bit is read. As soon as the parsed content matches a mapped code, we can decode it to the original symbol because there can not be a longer matching mapping. Due to the implementation via a Priority queue, which needs $O(log \: n)$ time per insertion and a tree with $n$ leaves has $2n-1$ nodes let this algorithm perform in $O(n \: log \: n)$ time where $n$ is the number of symbols.
}

\section{Other methods}
\label{ch:Principles of compression:sec:Other}
\par{
There are other well known methods like Move-to-Front coding or arithmetic coding also applied in different compression algorithms with different methods and therefore there own pros and cons. But other than Huffman coding, no further probability coding approach will be discussed in this scope. But its worth mentioning that real world applications are mostly combinations of methods, like Deflate\cite{deutsch1996rfc1951} which uses LZSS\cite{10.1145/322344.322346} (derivative of LZ77) and Huffman encoding or more recent ones like the PAQ family which is using a context mixing algorithm \cite{mahoney2009data}. Some combinations and PAQ implementations con be found in table \ref{tab:t20 stat of the art} where the state of the art compression methods and algorithms are discussed.
}

\subsection{Burrows-Wheeler-Transformation}
\label{ch:Principles of compression:sec:Other:subSec:bwt}

\par{
The Burrows-Wheeler-Transformation is not a compression technique but rather a method to prepare text to increase its compression potential, described by M. Burrows and D. J. Wheeler \cite{Burrows94}. It is a Transformation of a string $w$ of $n$ characters by forming the $n$ cyclic rotations of $w$, sorting them lexicographical, and extracting the last character of each of the rotations. Each rotation is induced by a righ shift of the original word. The right-shift of a word $w=a_1 ... a_n$ is dented as $r(w) = a_na_1...a_{n-1}$ and the $i$-fold right shift $r^i(w)$ is defined inductively by $r^0(w) = w$ and $r^{i+1}(w) = r(r^i(w))) $ which is also defined for $i \grq n$ as $r^i(w)=r^j(w)$ where $j = i \: mod \: n$. The result of this transformation is a string L, consisting of these last characters of each sorted rotation. The algorithm also has to add special termination symbols or compute the index $I$ of the original string S in the sorted list of rotations. Surprisingly, there is an efficient algorithm to invert the transformation back to the original string S given only L and the Index $I$ \cite{Burrows-linear-time}.
}

\par {
As an example of the creation of L, given the input string $w = crabab$, $n = 6$, and the alphabet $\Sigma = \{a,b,c,r\}$ with $a \textless b \textless c \textless r $. Create a $N \times N$ matrix $M$ whose elements are characters, and
whose rows are rotations of $w$. Sort all rotations in lexicographical order.
}

\par{
In this example, the index is $I = 4$ and the matrix $M$ is

\begin{table}[h]
	\centering
	\begin{tabular}{r|l|l|l|l|l|l}
		row 1 & a & b & a & b & c & r\\
		\hline
		row 2 & a & b & c & r & a & b\\
		\hline
		row 3 & b & a & b & c & r & a\\
		\hline
		row 4 & b & c & r & a & b & a\\
		\hline
		row 5 & c & r & a & b & a & b\\
		\hline
		row 6 & r & a & b & a & b & c
		\label{tab:t10 bwt-example}
	\end{tabular}
	\caption{Burrows Wheeler Transformation Matrix (all cyclic rotations)}
\end{table}
The resulting string $BWT(w)=L$ corresponds to the last column of $M$, with characters $M[0,n -1],\: \dots \: ,M[n - 1, n - 1]$. The output of the transformation is the pair $(L, i)$, in the example, $L = ‘rbaabc’$ and $i = 5$. Obviously the string $L$ contains consecutive identical characters, which results in better compressibility as shown in \ref{ch:Analysis:sec:Improvements by Preprocessing:subSec:bwt}. Longer words with more identical characters in general tend to have more equal characters in series.
}

\subsection{Inverse Burrows-Wheeler-Transformation}
\label{ch:Principles of compression:sec:Other:subSec:bwtInverse}
\par{
 The BWT is invertible and given $(BWT(w), i)$ where $i$ is the index of $w$ in $M(w)$ one can reconstruct the word $w$. There is a formal proof \cite{kufleitner2009bijective} available for this as well as the bijective variant described in the next section. To abstract the idea of inverting the permutation using only $L$ and $i$ is described by the following process by carrying on with the example. We now reconstruct the original word from the pair $(L,i)$ by first calculating the standard permutation of $L$ by writing $L$ column-wise, add positions, and then sort the pairs lexicographically.
 \begin{table}[h]
 	\centering
 	\begin{tabular}{l|l|l}
		word & word with position & sorted\\
		\hline
		r & (r,1) & (a,3) \\
		b & (b,2) & (a,4) \\
		a & (a,3) & (b,2) \\
		a & (a,4) & (b,5) \\
		b & (b,5) & (c,6) \\
		c & (c,6) & (r,1)
 		\label{tab:t11 standard permutation}
 	\end{tabular}
 	\caption{Standard permutation generation of the word $L$}
 \end{table}
}
\par{
This yields the standard permutation of:
\begin{gather}  
\pi_L = 
\begin{pmatrix} 1 & 2 & 3 & 4 & 5 & 6\\ 3 & 4 & 2 & 5 & 6 & 1 \end{pmatrix}
\end{gather}
Following $\pi^1_L(5)$ to $\pi^6_L(5)$ gives us a sequence of positions with $\pi_L(5)=6$ as a beginning, described by:
\begin{gather} 
6 \xrightarrow[]{\pi_L} 1 \xrightarrow[]{\pi_L} 3 \xrightarrow[]{\pi_L} 2 \xrightarrow[]{\pi_L} 4 \xrightarrow[]{\pi_L} 5
\end{gather}
If this sequence is applied to the labeling function of the word $L$ it results in 
\begin{gather}  
\lambda_L(6) \: \lambda_L(1) \: \lambda_L(3) \: \lambda_L(2) \: \lambda_L(4) \: \lambda_L(5)
= crabab = w
\end{gather}
This shows that it is possible to reconstruct the word $w$ given the transformation $BWT(w)$ ad the index $i$.
}

\subsection{Bijective Burrows-Wheeler-Scott Transformation}
\par{

}

%% ==============================
\section{State of the Art}
%% ==============================
\label{ch:Principles of compression:sec:SOTA}
\par{
The current state of the art at the time of writing is depicted in the table below, all algorithms used highest possible compression scheme available. In general there is always a balance between compressed size and compression or decompression speed, where the faster algorithms use mostly some form of dictionary coding sometimes in combination with a Huffman coder to be able to output variable length codes. The more advanced ones like PPMd or ZPAQ use complex context modeling or context mixing approaches where they generate a probability distribution for the next symbol based on just read symbols. In general these complex methods achieve better compression at the expense of using more space or time. The currently modern and most used algorithms have been executed with the Calgary corpus and the results are shown in table \ref{tab:t20 stat of the art}, all files were processed separately. Algorithms like \textit{compress}, \textit{gzip} and \textit{ZIP} use a dictionary based method like LZ77 or some derivative and achieve reasonable compression at very fast rates. \textit{bzip2} and \textit{p7zip} use a Burrows-Wheeler-Transformation (described in \ref{ch:Principles of compression:sec:Other:subSec:bwt}) and then a combination of different techniques to improve performance. They sometimes even offer explicit choosing of a method, like \textit{7zip} which achieves even better compression when using an advanced probabilistic method called Prediction by partial matching, also implemented but with even better results by \textit{ZPAQ} which performs even better while still maintaining reasonable speeds. In the last decade methods like \textins{zstandard} developed by Facebook or \textit{brotli} developed by Google arose, mostly implementing ANS and other methods arose but they are either desired to be just faster or designed for a different task like \textit{brotli} for HTML and JSON compression and do not achieved best results on this corpus. In the context of the \href{http://prize.hutter1.net/}{hutter prize}, a competitive compression challenge, other advanced probabilistic methods were developed like \textit{paq8hp1} to \textit{paq8hp12}, a series of algorithms by Alexander Rhatushnyak \cite{mahoney2011large} which won the challenge. A more recent fork of \textit{paq8} from 2014 is the current leader, \textit{cmix}. It is still the best performing algorithm on this list and achieved by far the best results on the Calgary corpus at expense of almost 3 hours of computing and 32GB of needed ram, displayed in \ref{tab:t100benchmark}. Unfortunately it was not possible to run the paq8hp* variants due to an major bug in the only available releases. Most algorithms are available from the default repositories of your Linux distribution, or in the case of Fedora and CentOS, pre-installed with your operating system. The paq8hp* algorithms can be found on the Hutter price website and cmix is developed on \href{https://github.com/byronknoll/cmix}{github}.
	\begin{table}[h]
		\begin{tabular}{r|r|r|r|r}
			method & options &  size in bytes & compression & \textit{bps}\\
			\hline
			uncompressed & & 3,145,718 & 100.0\% & 8.00 \\
			compress & & 1,250,382 & 40.4\% & 3.24 \\
			gzip v1.10 & -9 & 1,021,720 & 32.4\% & 2.60\\
			ZIP v3.0 &-9& 1,019,783 & 32.4\% & 2.59 \\
			zstandard 1.4.2& --ultra-23 -long=30 & 887,004 & 28.1\% & 2.25\\
			bzip2 v1.0.8 & --best & 832,443 & 26.4\% & 2.11 \\
			brotli & -q 11 -w 24 & 826,638 & 26.3\%& 2.10\\
			p7zip 16.02 (deflate) & a -mx10 & 821,873 & 26.1\% & 2.08 \\
			p7zip 16.02 (PPMd) & a -m=PPMd 0=32 & 763,067& 24.2\% & 1.93 \\
			ZPAQ v7.15 & -m5 & 659,700 & 20.9\% & 1.67  \\
			paq8hp* & - & - & - & - \\
			cmix v18 & -c -d & 554,983 & 17.6\% & 1.41 		
		\end{tabular}
				\label{tab:t20 stat of the art}
			\caption{State of the Art compression ratios}
	\end{table}
}

%% ==============================
\section{Limits of compression}
%% ==============================
\label{ch:Principles of compression:sec:Limits of Conpression}
\par{
All so far introduced methods and techniques work by removing redundant information but applied to random data without any structure or patterns they will fail to compress. This is explained by the Kolmogorov complexity\cite{kolmogorov} off a string $s$ which is the shortest possible program with $s$ as output. Consider the following two strings of equal length \textit{ababababab} and \textit{4c1j5b20fg}. While the first string can simply be denoted as $5 \times ab$, the other one has obviously no simple representation hence it has a higher Kolmogorov complexity. Using the pigeonhole principle it is also possible to trivially prove this by assuming $s$ is a binary string and $l(s) \: = \: n$. This implies $2^n$ possible string, all unequal. Assuming we can reduce all of them by 1 character, they only need $m = n - 1$ space and can only have $2^m$ possible unequal strings. This results in a contradiction because with lossless compression we need a distinct reverse operation and we cannot decode $2^{n-1}$ distinctly into $2^n$ strings. So to recap, no algorithm can compress all data of a given length, even by just one byte. If this was not true it would result in algorithm that could compress every string, so it could be applied recursively until any string has length 0. Therefore it is obvious that there exist some hard limits to compression.
}
%%% Local Variables: 
%%% mode: latex
%%% TeX-master: "thesis"
%%% End: 
