%% eval.tex
%% $Id: eval.tex 61 2012-05-03 13:58:03Z bless $

\chapter{Evaluation}
\label{ch:Evaluation}
%% ==============================
Moving on to the evaluation, we start by comparing the achieved results using different preprocessing options or combination of those. To start of it is clear that the best compression ratio was not accomplished by using a combination of different mapping techniques and a vertical interpretation of the input data but we now take a closer look at the discrepancy between each result. It is also clear that with the help of such methods, run length encoding can become a suitable compression algorithm for more than just pellet based images although it is clearly not as sophisticated as advanced state of the art compression methods mentioned in section \ref{ch:Principles of compression:sec:SOTA}.

%% ==============================
\section{Functional Evaluation}
%% ==============================
\label{ch:Evaluation:sec:Functional Evaluation}
\par{
The functional evaluation was performed in two steps. The first question is weather the algorithm works and no data is lost during the process of encoding. The decoder shows that this is the case and all information can be reconstructed, hence the suggested algorithm works as intended. The next question was, if it is just a method suited for this corpus and as it turned out, it performed even better on the newer and more frequently suggested Canterbury corpus than id did on the Calgary corpus. Also the abstinence of internal operations and large or complex data structures to hold all th input data or even collecting the values of same significance in memory into byte arrays greatly improved time performance of the algorithm described. Encoding is reasonable fast with measured 16 seconds but the decoding is  
}

\par{
	TODO
}
	\begin{table}[h]
	\centering
	\begin{tabular}{r|r|r|r|r}	
		file & size original & size encoded & ratio in \% & \textit{bps}\\
		\hline
alice29.txt & 152089 & 65445 & 43.03 & 3.44 \\
asyoulik.txt & 125179 & 59291 & 47.36 & 3.79 \\
cp.html & 24603 & 11073 & 45.01 & 3.60 \\
fields.c & 11150 & 5183 & 46.48 & 3.72 \\
grammar.lsp & 3721 & 1923 & 51.68 & 4.13 \\
kennedy.xls & 1029744 & 229823 & 22.32 & 1.79 \\
lcet10.txt & 426754 & 170593 & 39.97 & 3.20 \\
plrabn12.txt & 481861 & 215628 & 44.75 & 3.58 \\
ptt5 & 513216 & 82136 & 16.01 & 1.28 \\
sum & 38240 & 19616 & 51.30 & 4.10 \\
xargs.1 & 4227 & 2515 & 59.50 & 4.76 \\
		\hline
		all files & 2814880 & 867322 & 30.81 & 2.46
	\end{tabular}
	\caption{Canterbury encoded, all preprocessing steps, using Huffman encoding for all counted runs}
	\label{tab:t100:Canterbury encoded, all preprocessing steps, using Huffman encoding for all counted runs}
\end{table}
\ldots
%% ==============================
\section{Benchmarks}
%% ==============================
\label{ch:Evaluation:sec:Benchmarks}
- Benchmark with the Calgary corpus\\
- \url{http://www.data-compression.info/Corpora/} \\
\ldots
\par{
TODO: time over both encoding and decoding
}
	\begin{table}[h]
	\begin{tabular}{r|r|r|r|r|r}
		 & & & & time\\
		method  &  size in bytes & compression & \textit{bps}& encoding & decoding\\
		\hline
		uncompressed & 3,145,718 & 100.0\% & 8.00 &\\
		compress & 1,250,382 & 40.4\% & 3.24 & 0.039s\\
		modified vertical RLE & 1,237,380 & 39.3\%& 3.14 & 16.45s\\
		gzip v1.10 & 1,021,720 & 32.4\% & 2.60 & 0.232s\\
		ZIP v3.0 & 1,019,783 & 32.4\% & 2.59 & 0.214s\\
		zstandard 1.4.2& 887,004 & 28.1\% & 2.25 & 0.951s\\
		bzip2 v1.0.8 & 832,443 & 26.4\% & 2.11 & 0.191s\\
		brotli & 826,638 & 26.3\%& 2.10 & 4.609s\\
		p7zip 16.02 (deflate) &  821,873 & 26.1\% & 2.08 & 0.431s \\
		p7zip 16.02 (PPMd) &  763,067& 24.2\% & 1.93 & 0.345s\\
		ZPAQ v7.15 & 659.700 & 20.9\% & 1.67 & 7.452s \\
		paq8hp* & - & - & - & - \\ 
		cmix v18 & 554,983 & 17.6\% & 1.41 & <3h		
	\end{tabular}
	\label{tab:t100benchmark}
	\caption{Benchmark on the Calgary Corpus}
\end{table}
%% ==============================
\section{Conclusion}
%% ==============================
\label{ch:Evaluation:sec:Conclusion}

Am Ende sollten ggf. die wichtigsten Ergebnisse nochmal in \emph{einem} kurzen Absatz zusammengefasst werden.

%%% Local Variables: 
%%% mode: latex
%%% TeX-master: "thesis"
%%% End: 
