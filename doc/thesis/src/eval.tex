%% eval.tex
%% $Id: eval.tex 61 2012-05-03 13:58:03Z bless $

\chapter{Evaluation}
\label{ch:Evaluation}
%% ==============================
Moving on to the evaluation, we start by comparing the achieve results using different preprocessing options or combination of those. To start of it is clear that the best compression ratio was accomplished by using a combination of different techniques and a vertical interpretation of the input data but we now take a closer look at the discrepancy between each result. It is also clear that with the help of such methods, run length encoding can become a suitable compression algorithm for more than just pellet based images although it is clearly not as sophisticated as advanced state of the art compression methods mentioned in section \ref{ch:Principles of compression:sec:SOTA}.

%% ==============================
\section{Functional Evaluation}
%% ==============================
\label{ch:Evaluation:sec:Functional Evaluation}

\ldots
%% ==============================
\section{Benchmarks}
%% ==============================
\label{ch:Evaluation:sec:Benchmarks}
- Benchmark with the Galgary corpus\\
\url{http://www.data-compression.info/Corpora/} \\
\ldots

\subsection{Burrows-Wheeler-Transformation}

}


\subsection{Vertical encoding}
\par{


}

%% ==============================
\section{Conclusion}
%% ==============================
\label{ch:Evaluation:sec:Conclusion}

Am Ende sollten ggf. die wichtigsten Ergebnisse nochmal in \emph{einem} kurzen Absatz zusammengefasst werden.

%%% Local Variables: 
%%% mode: latex
%%% TeX-master: "thesis"
%%% End: 
