%% eval.tex
%% $Id: eval.tex 61 2012-05-03 13:58:03Z bless $

\chapter{Evaluation}
\label{ch:Evaluation}
%% ==============================
Hier erfolgt der Nachweis, dass das in Kapitel~\ref{ch:Entwurf}
entworfene Konzept funktioniert. 
Leistungsmessungen einer Implementierung werden immer gerne gesehen.

%% ==============================
\section{Functional Evaluation}
%% ==============================
\label{ch:Evaluation:sec:Functional Evaluation}
- Mathematical Comparison\\
- Comparison of encoded file sizes\\
- comparing to regular REL and Huffman coding\\
\ldots

%% ==============================
\section{Benchmarks}
%% ==============================
\label{ch:Evaluation:sec:Benchmarks}
- Benchmark with the Galgary corpus\\
\url{http://www.data-compression.info/Corpora/} \\
\ldots

%% ==============================
\section{Conclusion}
%% ==============================
\label{ch:Evaluation:sec:Conclusion}

Am Ende sollten ggf. die wichtigsten Ergebnisse nochmal in \emph{einem} kurzen Absatz zusammengefasst werden.

%%% Local Variables: 
%%% mode: latex
%%% TeX-master: "thesis"
%%% End: 
