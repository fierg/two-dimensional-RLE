\chapter{Implementation}
\label{ch:Implementation}
%% ==============================
All algorithms described have been implemented using Kotlin, because it can be compiled for the Java Virtual Machine as well as native, so it seemed like a good balance between a native implementation and higher language conciseness and fault-tolerance. Also there were some libraries available for Byte- and Bit-Operations on streams which proved to be quite useful, although there was almost no documentation available. This one and all other libraries are described in section \ref{ch:Conceptual Design:sec:Impl:subsec:libs}. The main focus is o the encoder and decoder classes but the other modules will be discussed as well.

\section{Binary and byte wise RLE}
\par{
	The simple binary and byte wise RLE are rather trivial and implemented in one encoder, the \textit{StringRunLengthEncoder}. The binary version is implemented with the mentioned BitStream from the IOStreams library. It allows working on a stream and reading and writing bit by bit and also reading the next n bit as signed or unsigned number which comes in handy during decoding. In general, it is called with a variable $b$ bitsPerRun which sets the used bits to store one run. At first a maximum run length is determined by the maximum value $b$ can store as a binary string, $l(b) = n$ implies a maximum value of $2^{n-1}$. Then the input is read consecutively in bits and the runs of equal bits are counted. We are always assuming the run starts with zero, if this is not the case a leading run of 0 is added, which means there are zero times 0 at the beginning. If a run exceeds the maximum run length, the maximum is written and again an artificial zero is added to the output stream to signal a length higher than the maximum. Each run can simply be written to the output stream with the desired amount of bits per run. During decoding, we assume the same $b$ and can then always read $n$ bits of the stream as unsigned value, know each run again and can therefore reconstruct the original data.
}
\par{
	Byte wise RLE is working a bit differently but with the same idea of counting $n$ runs of equal information. This time it is applied on a byte level, reading byte after byte. If the next byte is the same as the current one, the counter is incremented, if not the run and the byte value are encoded as pair $(n, byte)$ to the output. The byte value itself still needs 8 bit of information but the run does not. Most raw untreated data does not contain long runs of consecutive identical byte values average run length is rather small. This implied storing a count of 1 or 2 in 8 bits of space which in turn explained the expansion in size seen in table \ref{tab:t31 Byte-wise RLE on the Calgary Corpus}. Therefore this version was also implemented with an arbitrary amount of bits to save per run, to minimize the overhead. If a run exceeds the maximum, which is again determined by the amount of bits stored per run, it is encoded twice, once with the maximum count and once with the remaining count. We do not need a zero run in this version because we also store the value itself, therefore we can count without a zero. This means for an example run of 4 times the value \textit{0xFF} and 4 bits per run saved, it will be encoded as the pair $(0011,0xFF)$ or $001111111111$ as consecutive binary stream.
}
\section{Vertical binary RLE}
\par{
	The basic ideas described in sections \ref{ch:Analysis:sec:Improvements by Preprocessing:subSec:vertReading} and \ref{ch:Conceptual Design:sec:Parallel Byte Reading} where only a small variance compared to regular binary RLE. It is realized with the use of BitStreams again. This stream interface offers a position $p$, which corresponds to the byte value and a offset $o$, which is a bit value with significance $o$ of byte $p$, which allows reading all bits of the same significance in order. This was done for significance zero to seven to read all bits in a vertical manner as in the examples. Then each run is again counted with the same method including a maximum run length defined by the amount of bits used to count a run and the runs are then written with the fixed amount of bits to the output stream.
}
\par{
	TODO: Decoding
}
\section{Byte Remapping}
\par{
	TODO
}
\section{Burrows Wheeler Transformation}
\par{
	As mentioned earlier, the Burrows-Wheeler-Transformation is implemented in 3 different versions, starting of with the naive an unsophisticated one. It is implemented with the use of start and stop symbols (0x02 as STX and 0x03 as ETX) and with the creating and sorting of all cyclic rotations of the input string. This can be done for all text input files but not for binary data because files containing the start or stop symbols confused the algorithm and made the inverse transformation impossible. Additionally it is extremely slow due to its at least quadratic complexity, so another idea arose.
}
\par{
	TODO: second BWT impl
}

\par{
	TODO: sophisticated BWTS impl
}
\section{Huffman encoding}
\par{
TODO
}
\section{External libraries}
\label{ch:Conceptual Design:sec:Impl:subsec:libs}
\subsection{IOStreams}
- \href{https://discuss.kotlinlang.org/t/i-o-streams-for-kotlin/9802}{IOStreams for Kotlin}}
\subsection{libDivSufSort}
- \href{https://code.google.com/archive/p/libdivsufsort/}{libDivSufSort}\\
\subsection{libDivSufSort in Java}
- \href{https://github.com/flanglet/kanzi/releases}{libDivSufSort java impl}\\
\subsection{Others}
- static log\\
- kotlin coroutines\\
- jupiter \& junit5\\
- assertj\\
- google guava\\


%% ==============================
\section{Implementation Evaluation}
%% ==============================
\label{ch:Implementation:sec:Implementation Evaluation}
- evaluation of implementation choices made
\ldots
%%% Local Variables: 
%%% mode: latex
%%% TeX-master: "thesis"
%%% End: 
