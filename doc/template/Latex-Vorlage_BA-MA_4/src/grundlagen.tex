%% grundlagen.tex
%% $Id: grundlagen.tex 61 2012-05-03 13:58:03Z bless $
%%

\chapter{Basic principles}
\label{ch:Basic principles}
%% ==============================
Die Grundlagen müssen soweit beschrieben
werden, dass ein Leser das Problem und
die Problemlösung versteht, ohne weitere Literatur hinzuzuziehen.


%% ==============================
\section{Compression and Encoding fundamentals}
%% ==============================
\label{ch:Grundlagen:sec:Abschnitt1}

...

%% ==============================
\section{Run Length Encoding}
%% ==============================
\label{ch:Grundlagen:sec:Abschnitt2}

...

%% ==============================
\section{State of the Art}
%% ==============================
\label{ch:Grundlagen:sec:SOTA}
Die Literaturrecherche soll so vollständig wie möglich sein und bereits existierende relevante Ansätze (Verwandte Arbeiten / State of the Art / Stand der Technik) beschreiben bzw. kurz vorstellen.
Es soll aufgezeigt werden, wo diese Ansätze Defizite aufweisen oder nicht anwendbar sind, z.B. weil sie von anderen Umgebungen oder Voraussetzungen ausgehen.

Je nach Art der Abschlussarbeit kann es auch sinnvoll sein, diesen Abschnitt in die Einleitung zu integrieren oder als eigenes Kapitel aufzuführen.

Beispiel, wie mit LaTeX zitiert werden kann: \cite{TB98,JSAC96,qosr}

%%% Local Variables: 
%%% mode: latex
%%% TeX-master: "thesis"
%%% End: 
