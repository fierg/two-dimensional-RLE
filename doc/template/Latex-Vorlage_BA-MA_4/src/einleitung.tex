%% Einleitung.tex
%% $Id: einleitung.tex 61 2012-05-03 13:58:03Z bless $
%%

\chapter{Einleitung}
\label{ch:Einleitung}
%% ==============================
Die Einleitung besteht aus der Motivation, der Problemstellung, der Zielsetzung und einem erster Überblick über den Aufbau der Arbeit.

%% ==============================
\section{Motivation}
%% ==============================
\label{ch:Einleitung:sec:Motivation}

Warum ist das zu bearbeitende Themengebiet spannend und relevant?

- bezug auf historie von RLE\\
- heutige verwendung fax (mit huffman coding)\\
- schnelles decoding gut für client server arch\\

%% ==============================
\section{Problemstellung}
%% ==============================
\label{ch:Einleitung:sec:Problemstellung}

Welches Problem/welche Probleme können in diesem Themengebiet identifiziert werden?

- verbesserung der ursprünglichen impementation\\
- optimierung des decodings\\

%% ==============================
\section{Zielsetzung}
%% ==============================
\label{ch:Einleitung:sec:Zielsetzung}

Was ist das Ziel der Arbeit. Wie soll das Problem gelöst werden?

- bessere kompressionsrate im vergleich zu konventionellem RLE\\
- gleiche oder ähnliche decoding zeit ?\\
- ansatz beschreiben\\


%% ==============================
\section{Gliederung/Aufbau der Arbeit}
%% ==============================
\label{ch:Einleitung:sec:Gliederung}

Was enthalten die weiteren Kapitel? Wie ist die Arbeit aufgebaut? Welche Methodik wird verfolgt?


%%% Local Variables: 
%%% mode: latex
%%% TeX-master: "thesis"
%%% End: 
